\documentclass[a4paper,10pt]{article}
\usepackage[utf8]{inputenc}

%opening
\title{Notes for building free-formed surface classification algorithme}
\author{Ziqing Wu}

\begin{document}

\maketitle

% \begin{abstract}
% 
% \end{abstract}

\section{Definition of Classes}

For machining problems on 3+2 axis machine, the patterns that M.Redonnet have identified including (en français): plan, bosse, creux, marche, tuile, gorge, selle and méplat. According to these patterns, a set of ``standard surfaces'' has been created on a grid of 80*80. For each surface, a label is attributed as a string. 

\section{Data Augmentation}

\subsection{Unit Transformation}
The first obstacle that we encounter is that we don't have a dataset. So we need to generate a dataset to train the neural network. As the idea of data augmentation commonly used in computer vision, we can also apply some suited transformation on the free-formed surfaces. As the surfaces are generated by control points and defined by Bézier curves, and each set of control points corresponds to a unique surface. Transformations can be applied directly to the control points.

Here we list transformations created:
\begin{itemize}
 \item Translation
 \item Rotation
 \item Noise injection
 \item Scaling
\end{itemize}

It is recommended to implement an online algorithme for the data augmentation process, meaning that the transformations will be applied only when the data will be used for training the network.

\subsection{Pipeline}
To conduct the online algorithme, a pipeline is built with unit transformations. 

\subsection{Preparation of Test Dataset}
Instead of applying a general transformation pipeline to all the surfaces, one can propose some procedures to create one specific surface. For instance, one can generate a normal vector in the space then create 9 control points which are situated in the surface perpenticular to it. 


\section{Structure of CNN}

\section{Results of Classification}

\section{Extension to Object Detection with CNN}

Our classification problem is more interesting if we can do an analogy from the object detection problem in computer vision, meaning that instead of attributing one label to each surface, we can draw a box to cover the region where one specific character is shown. For instance for a surface combining plan, bosse and creux, we are hoping that the label can not only indicate the class name, but also indicate the positions of each identified class.


\end{document}
